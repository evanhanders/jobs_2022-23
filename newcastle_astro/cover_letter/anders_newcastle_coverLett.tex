\documentclass[12pt, a4paper]{letter}
\usepackage{graphicx}
\usepackage{fancyhdr}
\usepackage{geometry}
\usepackage{hyperref}

\usepackage[normalem]{ulem}
\usepackage{xcolor}
\newcommand{\edit}[2]{\textcolor{purple}{\sout{#1} \textbf{#2}}}

\geometry{tmargin=1in}
\geometry{bmargin=1in}
\geometry{lmargin=0.9in}
\geometry{rmargin=0.9in}
\geometry{headsep=1in}
\setlength\topmargin{-2.5in}
%\geometry{margin=1in}
%\geometry{headheight=0.5in}
%\fancypagestyle{firstpage}{\fancyhf{}
\fancypagestyle{firstpage}{\fancyhf{}
\setlength\topmargin{-1.25in}
\fancyhead[C]{\includegraphics[width=\paperwidth,keepaspectratio=true]{ciera_header.pdf}}
\fancyfoot[C]{\includegraphics[width=\paperwidth,keepaspectratio=true]{ciera_footer.png}}
\renewcommand{\headwidth}{\paperwidth}
\fancyheadoffset{\marginparwidth}
\fancyfootoffset{\marginparwidth}}
\renewcommand{\headrulewidth}{0pt}
% Address is commented out so I can use an electronic signature below
% If address is not commented out, then change \fancypagestyle{first} to
% \fancypagestyle{empty}
%\address{Nicholas Chapman \\
%         1212 Central Street APT 1S\\
%         Evanston, IL 60201}
%\signature{Nicholas Chapman} % commented out because I use png signature file

%They want to know:
%are you a good fit to the department? -- paragraph 2 addresses this.
%are you productive? -- cv addresses this?
%is your work going to have a big impact in the future?

\begin{document}
\begin{letter}{
        Dr.~Angela Dyson \& Faculty Search Committee \\
        School of Mathematics, Statistics and Physics \\
        Newcastle University}
\opening{Dear Committee:}

    I am applying for the position of Lecturer in Astrophysics at the Newcastle University School of Mathematics, Statistics and Physics.
    With this letter, please find my curriculum vitae.

    Observable stellar phenomena like magnetic dynamos and waves are generated by turbulent convection which is poorly understood despite decades of study.
    Stellar convection occurs in highly turbulent magnetized plasmas, is subject to rotational Coriolis forces, and involves flows which must traverse extreme stratification.
    An additional difficulty in understanding stellar convection comes in timescale discrepancies: the overturning timescale of convective flows is typically orders of magnitude shorter than the host star's evolutionary timescales.
    In my research, I investigate the behavior of convection subject to some of these complications using numerical simulations and use my results to improve convection models for use in stellar evolution software instruments.
    I have found that convection in density-stratified atmospheres exhibit the same fundamental heat transport scaling laws as incompressible, Boussinesq, Rayleigh-B\'{e}nard convection.
    I have learned how to disentangle the scaling of turbulence and rotational constraint in the complicated parameter space of rotating convection.
    I have extensively studied how fast convective flows interact with the (slowly evolving) background atmosphere.
    More recently, I have learned what determines the location of the interface between a convective region and a stably-stratified region, and used that knowledge to help explain discrepancies between models and observations of massive stars.

    \textbf{My top five research outputs}, ranked in order of importance, are the following papers:
    \begin{enumerate}
        \item Anders et al. 2022a, ApJ 926, 169 (“Convective Penetration Paper”)
        \item Anders et al. 2022b, ApJL 928, L10 (“Schwarzschild-Ledoux Paper”)
        \item Anders et al. 2019, ApJ 884, 65 (“Entropy Rain Paper”)
        \item Jermyn, Anders, et al. 2022, ApJ 929, 182 (“Convective Penetration Paper II”)
        \item Anders et al. 2018, PRF 3, 083501 (“Accelerated Evolution Paper”)
    \end{enumerate}

%    Last fall, I gathered with top researchers in my field at the “Probes of Transport in Stars” program at KITP. 
%    At this program, I learned that members of multiple communities (1D stellar modellers, observers, and 3D hydrodynamicists) are highly interested in convective boundary mixing and its applications. 
%    This is why I list my Convective Penetration Paper (\#1) as my most highly influential work, and why I propose to study convective boundary mixing during my time at Exeter (please refer to my Case for Support, Focus I). 
%    This interest has led to a follow-up paper (\#4) which demonstrates that our theory of convective penetration shows promise for resolving discrepancies between observations and models. 
%    The Schwarzschild-Ledoux Paper (\#2) also came out of this KITP meeting, brought together multiple fields, and addressed a fundamental question in stellar convection.
    My most influential work focuses on penetrative convection.
    Observations suggest that stellar structure models and 1D prescriptions of convection underestimate the size of convection zones.
    To match these observations, boundary mixing processes implemented in 1D stellar structure models have to be finely tuned and varied from one type of star to another.
    I have developed the first \emph{a-priori} theory which can explain these observations (paper \# 1), and follow-up work by Dr.~Adam Jermyn (paper \#4) suggests that this theory takes a step towards solving this decades-old problem.
    I discovered this process while running simulations in which I expected very little mixing at the convective boundary, but the convection consistently mixed and expanded well beyond its expected boundary.
    Using my previous work (e.g., \#5) on the long-term secular evolution of convection zones, I realized that I was witnessing a long-timescale relaxation process.
    Zahn and Roxburgh's work on penetrative convection in the 1970s-90s provided me with analytical descriptions of this process.
    I modified their theory to account for the effects of viscous dissipation, which cannot be neglected even in astrophysical plasmas with miniscule viscosities (per the zeroth law of turbulence).
    This theory agreed well with laminar and turbulent three-dimensional simulations using the \emph{Dedalus} code.
    I parameterized this theory for inclusion in 1D stellar structure models, and am collaborating with Dr.~Cole Johnston to implement this theoretical prescription self-consistently into the 1D \emph{MESA} software instrument to understand how penetrative convection affects massive star evolution.
    Our forthcoming suite of simulations will also take into account the results of paper \#2, in which I demonstrated for the first time using 3D numerical simulations the proper criterion for determining the convective boundary without penetration when the simulation is evolved for long timescales.
    Upon arriving at Newcastle, if possible, I would submit an off-cycle STFC grant proposal to perform state-of-the-art, 3D simulations of massive star core convection to continue to unravel this mystery.

    Earlier in my career, I studied the ``solar convective conundrum,'' which is a discrepancy between theory and simulations which predict deep large-scale ``giant cells," and observations which do not observe these features.
    In paper \#3, I tested the ``entropy rain'' hypothesis, which posits that the Sun’s luminosity is primarily transported by cold downflowing ``raindrops'' launched from the solar surface, rather than large, warm upwellings. 
    I developed a theory for the transport of luminosity by a single raindrop by expanding the theory of ``thermals'' (cold blobs which evolve into buoyant vortex rings) to account for atmospheric density stratification.
    Surprisingly, I found that cold downflows can carry the full solar luminosity in the deep convection zone.
    Furthermore, since thermals are a model of atmospheric convection frequently used in the geophysical community, this paper has attracted cross-disciplinary attention and discussions.

    Finally, in my Accelerated Evolution Paper (\#5), I created a tool which circumvents the discrepancy between evolutionary and dynamical timescales.
    I demonstrated that convective flows significantly change as the background stratification evolves.
    I devloped an alternative numerical scheme to traditional timestepping which used up to factor of ten fewer computational resources than traditional methods in our tests. 
    This study created foundational numerical tools that enabled the science in papers \#1 and \#2.

    Together, these works demonstrate my research's breadth and my approach to science.
    I decompose complex problems into manageable research questions.
    I study those questions using direct numerical simulations, which I perform using the flexible \emph{Dedalus} pseudospectral framework. 
    Then I use simulation results to learn something about stellar interiors, and synthesize that understanding in a way that is useful for communities (e.g., the 1D stellar modeling and observing community) which are much larger than the small community of hydrodynamical modelling.
    I also develop numerical tools and schemes to push the limits of state-of-the-art simulations.
    Regarding contributions: I did the bulk of the creative and scientific work for my first-author papers from above (e.g., I developed the experiment and simulation code, performed simulations, carried out analyses, and wrote the bulk of the manuscripts).
    I contributed to paper \#4 by helping translate my mixing prescription into MESA properly and in helping synthesize the results.
    These works were primarily funded by personal fellowship awards.


    \textbf{Teaching Statement:} Good teachers are not born, they are made through dedicated practice and effort.
    My teaching philosophy is rooted in having a growth mindset and acknowledging that I will always have room to grow into a better teacher.
    This belief fits in with my general outlook and my attempt to have a Growth mindset both for myself as a teacher and a scientist (we can learn from our mistakes and grow).
    In addition to practical classroom expertise, teachers should understanding modern teaching pedagogy and best educational practices.
    I have done this throughout my academic career through e.g., my participation in UCSC ISEE's Professional Development Program twice and through taking CIRTL's ``Introduction to Evidence-Based STEM Teaching'' MOOC course.
    
    Seeking out training in teaching pedagogy has in particular taught me the importance of active learning, scientific inquiry, skill-learning, and the creation of equitable and inclusive learning environments.
    Active learning techniques (e.g., Think-Pair-Share) increases student exam performance and should be included alongside lecturing in the classroom.
    Scientific inquiry is a process that we practice regularly as scientists but students rarely practice authentically in the classroom.
    Courses should therefore include projects where students participate in genuine scientific inquiry with opportunities to study self-motivated investigations.
    Furthermore, we must acknowledge that our students benefit both in academia and industry from developing a well-rounded set of scientist skills, such as communication, collaboration, persistence, organization, reflection.
    Finally, teachers must create equitable environments where diverse learners can succeed.
    This can be done in some rather explicit ways (e.g., a verbal welcome statement to broad identities), and can also be incorporated into assessments by providing multiple pathways for students to productively participate.

    Outside of the classroom, a Lecturer must support young scientists through mentorship, and I currently provide research mentorship to five graduate students.
    I have led and participated in projects on topics including  magnetoconvection, internally heated convection, and magnetic field instabilities (please refer to my publication list).
    In addition to being scientifically productive, these mentoring relationships have provided opportunities to practice my mentoring skills like active listening and advocating for my mentees in helping them network.
    It is exciting to hear that Newcastle has been awarded an STFC Data Intensive Science CDT, and I would be thrilled to work with students who will be hired as a result of the CDT 
    Fluid dynamical simulations produce large quantities of data, and it is often unclear how to manage, store, or statistically analyze those datasets properly, so my work is topical to this award.

    \textbf{Equity, Diversity, and Inclusion:} I am dedicated to providing a just, equitable and inclusive environment in my mentoring relationships, my classrooms, and my host institution.
    I have committed myself to developing the role of under-represented groups in physics and in developing evidence-based programs to target these groups.
    When I was as a PhD student, I served on the CU Boulder Astrophysics graduate admissions committee for two years, and I developed the first standardized rubric used in this process.
    An updated version of this rubric is still in use, and its implementation has measurably reduced bias in the admissions process and improved the diversity of admitted students.
    As a postdoctoral fellow at Northwestern, I have forged connections between the CIERA astrophysics institute and the Baxter Center for Science Education (BCSE), which has connections to many underserved high schools in the Chicago area.
    CIERA now facilitates a yearly ``Astronomy day'' as part of BCSE's summer scholars program, providing an authentic exoplanets inquiry activity to high school students from traditionally underrepresented backgrounds and teaching them about the career paths in astronomy.
    I chaired the K12 education and public outreach taskforce, served for a year as a core member of CIERA's JEDI (Justice, Equity, Diversity, Inclusion) committee, and now serve in the Climate Action Team which is working with an external contractor to implement a sociosystemic organizational development plan, including a departmental climate survey, this year.

    As a Lecturer at Newcastle, I will continue my work in JEDI initiatives.
    I will work to complete the tasks outlined in the action list of the School's Athena Swan bronze award (or a future award application), and I will participate in existing public outreach initiatives.
    I will also employ evidence-based practices for reducing differential attrition in my classrooms. 
    I am intereted in developing (or expanding if it already exists) a Maths, Stats, and Physics mentorship network.
    Individuals at all levels (undergraduate students, postgraduate students, postdocs, lecturers) can benefit from both peer and vertical mentoring structures, and these structures have been shown to improve retention of underrepresented groups in STEM.
    CIERA is currently implementing a similar network, so I have connections with expertise in designing and implementing such a network.



    \textbf{Summary: } I am particularly excited about this Lecturer position.
    The opportunity to join a rapidly-growing Physics group and to form long-lasting collaborations as part of a growing cohort is enticing.
    I envision the potential for productive collaboration between myself and current members of the School like Prof.~Rogers and Drs.~Wood, Guervilly, and Bushby on topics like the interactions between convection and stable regions, rotationally-constrained convection, stratified convection, and magnetoconvection.
    I bring knowledge of new research areas, e.g., multi-timescale processes, expertise with the open-source \emph{Dedalus} code, and also multi-scale processes such as my ``entropy rain'' research.
    \emph{Dedalus} has been used in a wide variety of fields, and I hope to use it to form cross-disciplinary collaborations with members of the Applied Maths research group on topics like biological fluid dynamics and with members of the JQC Durham-Newcastle on topics in quantum fluid dynamics.
    I value teaching and mentorship, and am an engaged member of my department community.
    For all these reasons, I am an ideal candidate for this post.

    In support of my application, please feel free to contact the following individuals:

    \footnotesize
    \begin{tabular}{ll}
        \textbf{Prof.~Benjamin P.~Brown}                        &\hspace{0.25in} \textbf{Prof.~Daniel Lecoanet}                     \\
        University of Colorado, Boulder                         &\hspace{0.25in} Northwestern University    \\
        \hspace{0.2in}Dept.~Astrophysical \& Planetary Sciences &\hspace{0.45in} Dept.~Engineering Sciences \& Applied Mathematics \\
        Tel: +1 (303) 653-2371                                    &\hspace{0.45in} CIERA                          \\
        Email: bpbrown@colorado.edu                             &\hspace{0.25in} Tel: +1 (608) 335-3950 \\
                                                                &\hspace{0.25in} Email: daniel.lecoanet@northwestern.edu             \\

        \textbf{Dr.~Matteo Cantiello}                          & \\
         Flatiron Insitute                                     & \\
         \hspace{0.2in}Center for Computational Astrophysics   & \\
         Princeton University                                  & \\
         \hspace{0.2in}Dept.~Astrophysical Sciences & \\
         Tel: +1 (805) 280-8175  & \\
         Email: mcantiello@flatironinstitute.org & \\
    \end{tabular}

    \normalsize

%    Other things I need to talk about:
%    \begin{enumerate}
%        \item Find all possible collaboration avenues in Physics (\url{https://www.ncl.ac.uk/maths-physics/research/physics/astrophysics/})
%        \item STFC CDT
%        \item Openness to collaboration with Research in quantum fluids and gases is carried out as part of the JQC (Joint Quantum Centre) Durham-Newcastle, a formal collaboration between Newcastle University and the University of Durham
%        \item Excitement about being part of a growing department.
%        \item express a genuine interest and commitment to developing the role of under-represented groups in physics, and an interest in establishing innovative, evidence-based programmes that will target these groups at all levels
%        \item $\times$ Your top 5 research outputs (accounting for career breaks) in which you have had a significant input. This may include journal publications and other forms of output relating to impact. For multi-author works, describe your contribution AND research funding applied for or received.
%            \begin{enumerate}
%                \item See narrative CV from Rutherford.
%            \end{enumerate}
%        \item $\times$ What constitutes excellent teaching to you (teaching statement).
%            \begin{enumerate}
%                \item Good teachers are made, not born -- practice and experience.
%                \item Knowledge of teaching pedagogy; point towards CIRTL and PDP.
%                \item Inclusive teaching, active learning
%                \item Growth mindset; seeking and accepting feedback and growing from it0
%                \item Teaching practical skills e.g., the STEM skills rubric (useful for any career).
%            \end{enumerate}
%        \item $\times$ Your past contributions to Equality, Diversity and Inclusion (EDI) and what your contributions to EDI at Newcastle will be
%            \begin{enumerate}
%                \item CU STARs
%                \item Admissions rubric
%                \item K12 taskforce / Baxter
%                \item JEDI and climate survey
%                \item Mentorship
%                \item Future contributions: Athena Swan (esp.~in hiring decisions / improving hiring practices), work to reduce differential attrition in my classrooms (physics identity, discussion of UREP, values affirmation) [plenty of room to grow within the school], continue public outreach efforts.
%            \end{enumerate}
%    \end{enumerate}
    
    If there are any other questions or concerns please do not hesitate to contact me.
    Thank you for your time and consideration.

\closing{Sincerely,}
\vspace{-0.9in}
\fromsig{\includegraphics[scale=0.2]{signature.png}}\\
\fromname{Evan H. Anders}
\end{letter}

\end{document}
