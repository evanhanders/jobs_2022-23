\documentclass[12pt]{letter}
\usepackage{graphicx}
\usepackage{fancyhdr}
\usepackage{geometry}

\usepackage[normalem]{ulem}
\usepackage{xcolor}
\newcommand{\edit}[2]{\textcolor{purple}{\sout{#1} \textbf{#2}}}

\geometry{tmargin=1in}
\geometry{bmargin=1in}
\geometry{lmargin=1in}
\geometry{rmargin=1in}
\geometry{headsep=1in}
\setlength\topmargin{-2.5in}
%\geometry{margin=1in}
%\geometry{headheight=0.5in}
%\fancypagestyle{firstpage}{\fancyhf{}
\fancypagestyle{firstpage}{\fancyhf{}
\setlength\topmargin{-1.25in}
\fancyhead[C]{\includegraphics[width=\paperwidth,keepaspectratio=true]{ciera_header.pdf}}
\fancyfoot[C]{\includegraphics[width=\paperwidth,keepaspectratio=true]{ciera_footer.png}}
\renewcommand{\headwidth}{\paperwidth}
\fancyheadoffset{\marginparwidth}
\fancyfootoffset{\marginparwidth}}
\renewcommand{\headrulewidth}{0pt}
% Address is commented out so I can use an electronic signature below
% If address is not commented out, then change \fancypagestyle{first} to
% \fancypagestyle{empty}
%\address{Nicholas Chapman \\
%         1212 Central Street APT 1S\\
%         Evanston, IL 60201}
%\signature{Nicholas Chapman} % commented out because I use png signature file

%They want to know:
%are you a good fit to the department? -- paragraph 2 addresses this.
%are you productive? -- cv addresses this?
%is your work going to have a big impact in the future?

\begin{document}
\begin{letter}{
               Prof.~Lars Bildsten \& KITP Postdoctoral Scholars Search Committee \\
               Kavli Institute for Theoretical Physics \\
               University of California, Santa Barbara
           }

\opening{Dear Committee:}

    I am applying for the positions of Postdoctoral Scholar in Theoretical Physics (AJO \# 22377) and Postdoctoral Scholar in Stellar Astrophysics (AJO \# 22378) at the Kavli Institute for Theoretical Physics.
    With this letter, please find my curriculum vitae, publication list, research statement, and letters from Prof.~Benjamin Brown, Prof.~Daniel Lecoanet \& Dr. Matteo Cantiello.

    Astrophysical convection is ubiquitous, but is poorly understood and therefore offers exciting research avenues.
    The well-studied problem of Rayleigh-B\'{e}nard convection (RBC) lacks many of the complexities of astrophysical convection.
    For example, stars and planets have stratified atmospheres, global rotation, and long-timescale transients.
    On the other hand, simulations of astrophysical convection which include radiation hydrodynamics are often so complex that they become like observations themselves and are again difficult to gain understanding from.
    My research sits between these extremes: I try to study comprehensible physics problems which include one or some of the nuances in astrophysical research which I then work to describe in detail.
    I have found that the same fundamental scaling laws in RBC appear in stratified convective flows.
    I have gained insight into how to vary fundamental control parameters to separate turbulence and rotational constraint in rotating convection.
    I have extensively studied the interactions of fast and slow convective processes, paralleling the fast convection and slow evolution seen in stars.

    My most influential work focuses on penetrative convection.
    Observations suggest that stellar structure models and 1D prescriptions of convection underestimate the size of convection zones.
    To match these observations, boundary mixing processes implemented in 1D stellar structure models have to be finely tuned and varied from one type of star to another.
    I have developed the first \emph{a-priori} theory which can explain these observations, and follow-up work by Dr.~Adam Jermyn suggests that this theory takes a step towards solving this decades-old problem.
    I discovered this process while running simulations in which I expected very little mixing at the convective boundary.
    Significantly, the convection zone advanced well beyond the expected boundary.
    Using my previous work on the long-term secular evolution of convection zones, I realized that I was witnessing a long-timescale relaxation process.
    Zahn and Roxburgh's work on penetrative convection in the 1980s and 90s provided me with analytical descriptions of this process.
    I modified their theory to account for the effects of viscous dissipation, which cannot be neglected even in the turbulent regime of astrophysical flows (per the zeroth law of turbulence).
    This theory agreed well with laminar and turbulent three-dimensional simulations using the \emph{Dedalus} code.
    I parameterized this theory for inclusion in 1D stellar structure models.
    I am currently collaborating with Dr.~Cole Johnston to implement this theoretical prescription into \emph{MESA} to understand how this process affects stellar evolution.
    In my attached research statement, I discuss follow-up work that I want to pursue.

    Throughout my academic career, I have striven not only to conduct state-of-the-art research, but also to develop my pedagogical and mentoring expertise through workshop attendance and practice.
    I now provide research mentorship to five graduate students across two institutions (Northwestern and Univ.~Colorado).
    All of these mentorship relationships have led to collaborative contributions on papers which are published or in prep (marked on my publication list).
    Within my research mentoring relationships and more broadly within my academic institution, I am dedicated to providing a just, equitable and inclusive research environment.
    During my postdoctoral fellowship at CIERA, I chaired the K12 education and public outreach taskforce, I served as a core member of CIERA's JEDI (Justice, Equity, Diversity, Inclusion) committee, and I now serve on the Climate Action Team which is helping implement a sociosystemic organizational development plan, including a departmental climate survey, with the help of Visceral Change.
    I would be excited to participate in opportunities to mentor graduate and undergraduate students at KITP, to continue my record of departmental service, and to get involved with public outreach programs like CSEP's SST.

    KITP is the ideal host institute for me to carry out the next stage of my academic career.
    My expertise on convection and convective boundary mixing---combined with the expertise of Prof.~Lars Bildsten on stellar structure, stellar evolution, and opacity-driven convection---will ensure the success of my research plan.
    I also look forward to forming collaborations with Prof.~Omer Blaes to study MHD instabilities including the Tayler instability, and with Prof.~Timothy Brandt on questions in stellar evolution.
    I am an expert at using the \emph{Dedalus} code, and I hope to use it to expand my research horizons through collaborations outside of my past expertise.
    I would also be happy to help others at KITP learn how to apply this tool to their own research in fields including biophysics or quantum fluids.
    On a personal note, the recent TRANSTAR21 program was a transformative experience for me after the isolation of COVID-19, and this experience cemented KITP as the best possible place for me to be at this stage in my career.
    I'm excited to participate in future programs, in particular the ``Turbulence in Astrophysical Environments'' program in early 2024 and the ``Interconnections between the Physics of Plasmas and Self-gravitating Systems'' program in summer 2024.
    For all these reasons, I am an ideal candidate to be a KITP Postdoctoral Scholar.


    If there are any other questions or concerns please do not hesitate to contact me.
    Thank you for your time and consideration.

\closing{Sincerely,}
\vspace{-0.9in}
\fromsig{\includegraphics[scale=0.2]{signature.png}}\\
\fromname{Evan H. Anders}
\end{letter}

\end{document}
