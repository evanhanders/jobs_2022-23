\documentclass[12pt]{article}
%%%%%%begin preamble
\usepackage[hmargin=1in, vmargin=1in]{geometry} % Margins
\usepackage{hyperref}
\usepackage{url}
\usepackage[numbers]{natbib}
\usepackage{graphicx}
\usepackage{amsmath}
\usepackage{amsfonts}
\usepackage{amssymb}
\usepackage{wrapfig}

\usepackage{multicol}
\usepackage{etoolbox}
%\patchcmd{\thebibliography}{\section*{\refname}}
%    {\begin{multicols}{2}[\section*{\refname}]}{}{}
%\patchcmd{\endthebibliography}{\endlist}{\endlist\end{multicols}}{}{}


\usepackage[normalem]{ulem}
\usepackage{xcolor}
\newcommand{\edit}[2]{\textcolor{purple}{\sout{#1} \textbf{#2}}}

\hypersetup{
  colorlinks   = true,
  %citecolor    = blue
  citecolor    = blue
  % gray is not being found!?!
  % gray is found if pdfpages is used... crap.
  %citecolor    = grey
  %citecolor    = Gray
}


%% headers
\usepackage{fancyhdr}
\pagestyle{fancy}
\fancyhf{} % sets both header and footer to nothing
\lhead{Evan H. Anders}
\rhead{Publication Explanation}
\cfoot{\footnotesize{\thepage}}
%\pagestyle{empty}
%\pagenumbering{gobble}
%\renewcommand*{\thefootnote}{\fnsymbol{footnote}}

\renewcommand{\vec}{\ensuremath{\boldsymbol}}
\newcommand{\dedalus}{\href{http://dedalus-project.org}{Dedalus}}
\newcommand{\del}{\ensuremath{\vec{\nabla}}}
\newcommand{\scrS}{\ensuremath{\mathcal{S}}}

\newcommand{\prf}{PRF}
\newcommand{\prr}{PRR}
\newcommand{\ssr}{SSR}
\newcommand{\araa}{ARAA}
\newcommand{\mnras}{MNRAS}
\newcommand{\aap}{A\&A}
\newcommand{\apjl}{ApJL}
\newcommand{\apj}{ApJ}
\newcommand{\apjs}{ApJL}

\newcommand{\sct}[1]{\vspace{0.3cm}\hspace{-\parindent}\textbf{\underline{#1}}\hspace{0.3cm}}

%\newcommand{\nosection}[1]{%
%  \refstepcounter{section}%
%  \addcontentsline{toc}{section}{\protect\numberline{\thesection}#1}%
%  \markright{#1}}
%\newcommand{\nosubsection}[1]{%
%  \refstepcounter{subsection}%
%  \addcontentsline{toc}{subsection}{\protect\numberline{\thesubsection}#1}%
%  \markright{#1}}

%\usepackage{atbegshi}
%%%%%%end preamble


%Make bibliography 2col
\bibliographystyle{apj_small}
\makeatletter
\renewenvironment{thebibliography}[1]
     {\begin{multicols}{2}[\paragraph*{\refname}\vspace{-0.1in}]%
      \@mkboth{\MakeUppercase\refname}{\MakeUppercase\refname}%
      \list{\@biblabel{\@arabic\c@enumiv}}%
           {\settowidth\labelwidth{\@biblabel{#1}}%
            \leftmargin\labelwidth
            \advance\leftmargin\labelsep
            \@openbib@code
            \usecounter{enumiv}%
            \let\p@enumiv\@empty
            \renewcommand\theenumiv{\@arabic\c@enumiv}}%
      \setlength{\itemsep}{-2pt}
      \sloppy
      \clubpenalty4000
      \@clubpenalty \clubpenalty
      \widowpenalty4000%
      \sfcode`\.\@m}
     {\def\@noitemerr
       {\@latex@warning{Empty `thebibliography' environment}}%
      \endlist\end{multicols}}
\makeatother



\begin{document}
\thispagestyle{fancy}

I chose to include my recent article, ``Stellar Convective Penetration: Parameterized Theory and Dynamical Simulations'', with my application.
%This paper exemplifies my achievements and potential in the following ways.
This paper connects theoretical arguments (section 3) to measurable results from simulations (section 5) and then applies those results to stellar structure models and discusses observational consequences (section 6).
My research method is exemplified by this structure: I develop theoretical explanations, I test those explanations using carefully designed and minimal experiments, and then I use that theory to say something useful for the broader astrophysical community.
Please also note that this paper shows my willingness to try creative approaches for communicating my science even in traditional forms of media such as peer-reviewed journal articles.
We knew that this paper was long when we wrote it; in order to ensure that the important take-home messages are clear, I added a section after the introduction clearly stating the paper's central results.
This section ensures that our important conclusions are not lost deep in the results section, and also primes the reader with a framework for what to look for throughout the rest of the work.

\end{document}
