\documentclass[11pt]{article}
%%%%%%begin preamble
\usepackage[hmargin=0.85in, vmargin=0.7in]{geometry} % Margins
\usepackage{hyperref}
\usepackage{url}
\usepackage[numbers]{natbib}
\usepackage{graphicx}
\usepackage{amsmath}
\usepackage{amsfonts}
\usepackage{amssymb}
\usepackage{wrapfig}

\usepackage{multicol}
\usepackage{etoolbox}
%\patchcmd{\thebibliography}{\section*{\refname}}
%    {\begin{multicols}{2}[\section*{\refname}]}{}{}
%\patchcmd{\endthebibliography}{\endlist}{\endlist\end{multicols}}{}{}


\usepackage[normalem]{ulem}
\usepackage{xcolor}
\newcommand{\edit}[2]{\textcolor{purple}{\sout{#1} \textbf{#2}}}

\hypersetup{
  colorlinks   = true,
  %citecolor    = blue
  citecolor    = blue
  % gray is not being found!?!
  % gray is found if pdfpages is used... crap.
  %citecolor    = grey
  %citecolor    = Gray
}


%% headers
\usepackage{fancyhdr}
\pagestyle{fancy}
\fancyhf{} % sets both header and footer to nothing
\lhead{Evan H. Anders}
\rhead{Equity, Diversity, \& Inclusion Statement}
\cfoot{\footnotesize{\thepage}}
%\pagestyle{empty}
%\pagenumbering{gobble}
%\renewcommand*{\thefootnote}{\fnsymbol{footnote}}

\renewcommand{\vec}{\ensuremath{\boldsymbol}}
\newcommand{\dedalus}{\href{http://dedalus-project.org}{Dedalus}}
\newcommand{\del}{\ensuremath{\vec{\nabla}}}
\newcommand{\scrS}{\ensuremath{\mathcal{S}}}

\newcommand{\prf}{Physical Review Fluids}
\newcommand{\ssr}{Space Science Reviews}
\newcommand{\araa}{Annual Reviews of Astronomy and Astrophysics}
\newcommand{\mnras}{Monthly Notices of the Royal Astronomical Society}
\newcommand{\aap}{Astronomy \& Astrophysics}
\newcommand{\apjl}{The Astrophysical Journal Letters}
\newcommand{\apj}{The Astrophysical Journal}

%\newcommand{\nosection}[1]{%
%  \refstepcounter{section}%
%  \addcontentsline{toc}{section}{\protect\numberline{\thesection}#1}%
%  \markright{#1}}
%\newcommand{\nosubsection}[1]{%
%  \refstepcounter{subsection}%
%  \addcontentsline{toc}{subsection}{\protect\numberline{\thesubsection}#1}%
%  \markright{#1}}

%\usepackage{atbegshi}
%%%%%%end preamble


%Make bibliography 2col
\bibliographystyle{apj_small}
\makeatletter
\renewenvironment{thebibliography}[1]
     {\begin{multicols}{2}[\paragraph*{\refname}\vspace{-0.1in}]%
      \@mkboth{\MakeUppercase\refname}{\MakeUppercase\refname}%
      \list{\@biblabel{\@arabic\c@enumiv}}%
           {\settowidth\labelwidth{\@biblabel{#1}}%
            \leftmargin\labelwidth
            \advance\leftmargin\labelsep
            \@openbib@code
            \usecounter{enumiv}%
            \let\p@enumiv\@empty
            \renewcommand\theenumiv{\@arabic\c@enumiv}}%
      \setlength{\itemsep}{-2pt}
      \sloppy
      \clubpenalty4000
      \@clubpenalty \clubpenalty
      \widowpenalty4000%
      \sfcode`\.\@m}
     {\def\@noitemerr
       {\@latex@warning{Empty `thebibliography' environment}}%
      \endlist\end{multicols}}
\makeatother



\begin{document}
\thispagestyle{fancy}

\vspace{-0.2cm}
STEM has a demographics crisis which is particularly dire in the physical sciences. 
Our fields are more male and more white than the U.S.~population as a whole, which means that these fields are missing out on diverse perspectives and talent \citep{stem_laborforce_2021}. 
Identity-based inequities are further ingrained in the disparate nature in which we distribute credit in the form of citations \citep{teich_etal_2022}.
These inequities are immoral, and it is the work of all members of the STEM, Physics, and Astrophysics community---especially white men like myself, who are an overrepresented majority with excess power---to make the field more welcoming and inclusive.
There is a high differential attrition rate of underrepresented groups in baccalaureate and post-graduate programs \citep{whitcomb_singh_2021}, and university professors can make immediate impacts at these levels.

My focus on improving retention of underrepresented groups is rooted in social cognitive career theory \citep{kelly_2016}, which has three axes: ``physics self-concept and self-efficacy'', ``expectancy-value and planned behavior'', and ``motivation and self-determination''.

Physics self-concept and self-efficacy negatively manifests as feelings of inadequacy, lack of social support, and communication anxiety.
To combat this, we need to build young peoples' identities as scientists by providing opportunities to build competance, to perform science skills, and to receive recognition  \citep{hazari_etal_2010}.
In addition to building knowledge and competence, STEM education (in classrooms and research groups) must give students experience to \emph{perform} as scientists by e.g., giving talks or collaborating on projects \emph{in low-stakes environments}.
Furthermore, students deserve positive recognition for their accomplishments, but often only receive criticism (even if constructive) on a regular basis.
As a research mentor, I frequently encourage and complement my students for their work. 
Even small victories deserve to be celebrated, and I try to instill that mindset in my students.

Expectancy-value and planned behavior manifests as buying into negative stereotypes about oneself or experiencing negative environments.
Studies have shown that stereotype threat can be mitigated using values affirmation exercises \citep{miyake_etal_2010}, which I plan to use before tests in my classrooms.
To create inclusive classrooms, I practice small actions like reading an explicit welcome statement at the beginning of the term, asking students what they prefer I call them and what pronouns they prefer I use, etc.
I also try to design assessments with multiple ways to productively participate and succeed.
I still have a lot to learn, and plan to take the Inclusive STEM Project MOOC course in the coming year.

Motivation and self-determination can drive students out of STEM because they do not find it intrinsically motivating or socially relevant, or because they lack encouragement.
One way to build relevance is by employing lesson plans which allow students to engage with their culture \emph{and} science simultaneously.
One example of such a lesson plan--which was developed for middle school but is applicable to introductory undergraduate astronomy classes--can be found online here: \url{https://www.openscied.org/instructional-materials/8-4-earth-in-space/}.
In my mentoring relationships, I practice active listening \citep{jahromi_etal_2016} with my students to try to understand what motivates them so that I do not steer them out of intrinsically motivating questions.
%I try hard to provide guidance without “taking over” my students' projects so that they have the space to participate in scientific practices in a safe and judgment-free environment so they can succeed in higher-pressure environments (e.g., conference talks).

I have a track record of making my departments more just, equitable, diverse, and inclusive spaces.
As a graduate student, I created the first rubric used on the graduated admissions committee, iterated on that rubric, and led the push to adopt rubrics in this process permanently.
This rubric has measurably reduced the bias of reviewers which has led to the admission of more diverse classes.
As a postdoc at Northwestern, I have been an integral part of the push to hire Visceral Change to perform a climate survey this academic year.
I aim to use my power and privilege in the field to make systemic changes within my own departments to improve outcomes for all scientists regardless of their identity.
I look forward to leading similar initiatives at the University of Oregon or discussing other ways to improve the equity, diversity, and inclusivity of the Department of Physics in formal ways.

%
%\begin{enumerate}
%    \item Discuss social cognitive career theory and Kelly 2016 PRPER 12 020116. Talk about how you will work to bolster each axis in:
%        \begin{enumerate}
%            \item Physics self-concept and self-efficacy (pervasive feelings of inadequacy, lack of social support, communication anxiety)
%            \item Expectancy-value and planned behavior (negative stereotypes of women, undesirable clasroom environments, lack of female role models)
%            \item Motivation and self-determination (intrinsic appeal of physics, social relevance of physics, lack of encouragement)
%        \end{enumerate}
%    \item Cite Hazari et al 2010 JRST 47 8: ``The theoretical framework focuses on physics identity and includes the dimensions of student performance, competence, recognition by others, and interest."
%    \item Cite Hazari et al 2013 PRST-PER 9 020115: ``However, discussions about women’s underrepresentation have a significant positive effect."
%    \item Cite Miyake et al 2010 Science 330: ``Values affirmation reduced the male-female performance and learning difference substantially and elevated women’s modal grades from the C to B range."
%    \item Discuss and link explicit welcome statement
%    \item Discuss and link to manhattanhenge:
%        \begin{enumerate}
%            \item Astronomy is perceived as a particularly Western and White science (Ali 2010).
%            \item Higher attrition of women / BIPOC from astronomy/physics (Porter \& Ivie 2019).
%            \item Build relevance by centering cultural sustainability in design decisions. E.g., \url{https://www.openscied.org/instructional-materials/8-4-earth-in-space/}
%        \end{enumerate}
%\end{enumerate}

\vspace{-0.4cm}
{\scriptsize
\bibliography{biblio}
}
\end{document}
