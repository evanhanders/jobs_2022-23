\documentclass[11pt]{article}
%%%%%%begin preamble
\usepackage[hmargin=1in, vmargin=1in]{geometry} % Margins
\usepackage{hyperref}
\usepackage{url}
\usepackage{natbib}
\usepackage{graphicx}
\usepackage{amsmath}
\usepackage{amsfonts}
\usepackage{amssymb}
\usepackage{wrapfig}

\usepackage{multicol}
\usepackage{etoolbox}
%\patchcmd{\thebibliography}{\section*{\refname}}
%    {\begin{multicols}{2}[\section*{\refname}]}{}{}
%\patchcmd{\endthebibliography}{\endlist}{\endlist\end{multicols}}{}{}


\usepackage[normalem]{ulem}
\usepackage{xcolor}
\newcommand{\edit}[2]{\textcolor{purple}{\sout{#1} \textbf{#2}}}

\hypersetup{
  colorlinks   = true,
  %citecolor    = blue
  citecolor    = blue
  % gray is not being found!?!
  % gray is found if pdfpages is used... crap.
  %citecolor    = grey
  %citecolor    = Gray
}


%% headers
\usepackage{fancyhdr}
\pagestyle{fancy}
\fancyhf{} % sets both header and footer to nothing
\lhead{Evan H. Anders}
\rhead{Equity, Diversity, \& Inclusion Statement}
\cfoot{\footnotesize{\thepage}}
%\pagestyle{empty}
%\pagenumbering{gobble}
%\renewcommand*{\thefootnote}{\fnsymbol{footnote}}

\renewcommand{\vec}{\ensuremath{\boldsymbol}}
\newcommand{\dedalus}{\href{http://dedalus-project.org}{Dedalus}}
\newcommand{\del}{\ensuremath{\vec{\nabla}}}
\newcommand{\scrS}{\ensuremath{\mathcal{S}}}

\newcommand{\prf}{Physical Review Fluids}
\newcommand{\ssr}{Space Science Reviews}
\newcommand{\araa}{Annual Reviews of Astronomy and Astrophysics}
\newcommand{\mnras}{Monthly Notices of the Royal Astronomical Society}
\newcommand{\aap}{Astronomy \& Astrophysics}
\newcommand{\apjl}{The Astrophysical Journal Letters}
\newcommand{\apj}{The Astrophysical Journal}

%\newcommand{\nosection}[1]{%
%  \refstepcounter{section}%
%  \addcontentsline{toc}{section}{\protect\numberline{\thesection}#1}%
%  \markright{#1}}
%\newcommand{\nosubsection}[1]{%
%  \refstepcounter{subsection}%
%  \addcontentsline{toc}{subsection}{\protect\numberline{\thesubsection}#1}%
%  \markright{#1}}

%\usepackage{atbegshi}
%%%%%%end preamble


%Make bibliography 2col
\bibliographystyle{apj_small}
\makeatletter
\renewenvironment{thebibliography}[1]
     {\begin{multicols}{2}[\paragraph*{\refname}\vspace{-0.1in}]%
      \@mkboth{\MakeUppercase\refname}{\MakeUppercase\refname}%
      \list{\@biblabel{\@arabic\c@enumiv}}%
           {\settowidth\labelwidth{\@biblabel{#1}}%
            \leftmargin\labelwidth
            \advance\leftmargin\labelsep
            \@openbib@code
            \usecounter{enumiv}%
            \let\p@enumiv\@empty
            \renewcommand\theenumiv{\@arabic\c@enumiv}}%
      \setlength{\itemsep}{-2pt}
      \sloppy
      \clubpenalty4000
      \@clubpenalty \clubpenalty
      \widowpenalty4000%
      \sfcode`\.\@m}
     {\def\@noitemerr
       {\@latex@warning{Empty `thebibliography' environment}}%
      \endlist\end{multicols}}
\makeatother



\begin{document}
\thispagestyle{fancy}


STEM fields, particularly those in the physical sciences, have a demographics crisis. 
Our fields are more male and more white than the U.S.~population as a whole, which means that these fields are missing out on diverse perspectives and talent (cite). 
These inequities are further ingrained in the disparate nature in which we distribute credit in the form of citations (cite).
These inequities are immoral, and it is the work of all members of the STEM, Physics, and Astrophysics community---especially white men like myself, who make up the majority in these fields and have unequal power---to create a more inclusive and welcoming space.
A great deal of focus is rightly directed toward the “leaky pipeline” from grade school to college.
However, there is a high differential attrition rate of underrepresented groups in baccalaureate and post-graduate programs (cite), and university professors can make immediate impacts at these levels.

I try to root my EDI work in the physics education and sociology literature.
My focus on improving retention of these underrepresented groups is rooted in social cognitive career theory (cite); this theory has three axes: \emph{physics self-concept and self-efficacy}, \emph{expectancy-value and planned behavior}, and \emph{motivation and self-determination}.

The first of these areas can be addressed by helping build our students' identities as people in STEM.
This is achieved by providing opportunities for students to develop competence in the field (e.g., traditional education), opportunities to gain experience performing science skills (e.g., through giving talks, collaborating on a project, or writing papers), and opportunities to receive genuine and organic recognition for their accomplishments (e.g., from peer interactions or in the form of more open-ended inquiry activities).
In particular, I aim to help improve the physics self-concept and self-efficacy of underrepresented students to help them build identities as people in STEM which has 

The second axis occurs when students buy in to negative stereotypes about themselves (stereotype threat, cite) or are subject to undesireable classroom environment.
Studies have shown that stereotype threat can be mitigated using values affirmation exercises (cite), which I plan to use before tests or other high-stress environments in my classrooms.
I also strive to create inclusive classroom environments where people can come into the classroom as full people who experience the intersection of all of their identities and find success regardless.
To do this, I do small things (e.g., read an explicit welcome statement at the beginning of the term), and I also try to design my assessments with multiple ways to productively participate so that many different kinds of learners can succeed. (ehh)
I still have a long way to go in learning how to create a welcoming and inclusive space for all, but I plan to continue to build on my knowledge here, e.g., through taking the Inclusive STEM Project MOOC course in the coming year, taking further bystander intervention training, etc.

The third axis, motivation and self-determination, relates to an individual's desire to continue in a STEM career because they do (or do not) find it intrinsically motivating.
Astronomy in particular is perceived as a particularly western and white science (cite).
One way to build relevance is by creating assignments which allow students to engage with their own heritage and culture which can in turn peak their interest in scientific applications ().
(for an example of this designed at the middle school level but applicable to general astronomy courses at the college level, see URL).

%My experience with the CU-STARs group at the University of Colorado helped me see this competence-performance-recognition triad in action. 
%CU-STARs is a welcoming group for undergraduate students studying physics and astrophysics at CU. 
%This group has regular meetings in which graduate student leaders mentor the undergraduates and educate undergraduate students about common problems faced by young scientists like stereotype threat (cite), imposter’s syndrome (cite), etc.~while offering them ways to build themselves up like values affirmation (cite) and discussions of fixed vs.~growth mindset (cite). 
%In addition to these mentorship aspects, CU-STARs led outreach trips to underserved high schools across Colorado in which the undergraduates served as near-peer teachers to the high school students. 
%Through this the STARs have an opportunity to gain competence in both teaching and an area of astrophysics as they develop lesson plans, they have the chance to perform as scientists by teaching high schoolers and serving as a contact in science, and they have a chance to be recognized by us, the grad student leaders, as we gave them honest praise and constructive feedback to improve their lessons and teaching over time.

Beyond the classroom, I strive to implement this framework in my own research mentoring. 
Currently, I am a secondary mentor to five graduate researchers.
While doing research naturally builds scientific competence, I practice active listening (cite) with my students to ensure that I offer them advice or pathways forward based on where they're actually stuck rather than my initial perception of their problem (and these are often different).
%Many of these students preface questions with “this is a stupid question,” etc, from their years of being in classes where teachers are only seeking the “right” answer and I have to remind them that asking questions is how we all learn and build competence, and I try to encourage them to ask questions. 
I try hard to provide guidance without “taking over” my students' projects so that they have the space to participate in scientific practices in a safe and judgment-free environment so they can succeed in higher-pressure environments (e.g., conference talks).
And, importantly, I frequently encourage and complement my students for their work. 
Even small victories deserve to be celebrated, and I try to instill that mindset in my students.

I have a track record of striving to make my departments a more just, equitable, diverse, and inclusive space.
As a graduate student, I created the first rubric used on the graduated admissions committee, iterated on that rubric, and led the push to eventually adopt rubrics in this process permanently.
Adoption of this rubric measurably reduced the bias of reviewers (e.g., when they were tired compared to not), and has led to the admission of more diverse classes as a whole.
As a postdoc at Northwestern, I have been an integral part of the push to hire Visceral Change to perform a climate survey this academic year.
I aim to use my power and privilege in the field to make systemic changes within my own departments to improve outcomes for all scientists regardless of their identity.


\begin{enumerate}
    \item Discuss social cognitive career theory and Kelly 2016 PRPER 12 020116. Talk about how you will work to bolster each axis in:
        \begin{enumerate}
            \item Physics self-concept and self-efficacy (pervasive feelings of inadequacy, lack of social support, communication anxiety)
            \item Expectancy-value and planned behavior (negative stereotypes of women, undesirable clasroom environments, lack of female role models)
            \item Motivation and self-determination (intrinsic appeal of physics, social relevance of physics, lack of encouragement)
        \end{enumerate}
    \item Cite Hazari et al 2010 JRST 47 8: ``The theoretical framework focuses on physics identity and includes the dimensions of student performance, competence, recognition by others, and interest."
    \item Cite Hazari et al 2013 PRST-PER 9 020115: ``However, discussions about women’s underrepresentation have a significant positive effect."
    \item Cite Miyake et al 2010 Science 330: ``Values affirmation reduced the male-female performance and learning difference substantially and elevated women’s modal grades from the C to B range."
    \item Discuss and link explicit welcome statement
    \item Discuss and link to manhattanhenge:
        \begin{enumerate}
            \item Astronomy is perceived as a particularly Western and White science (Ali 2010).
            \item Higher attrition of women / BIPOC from astronomy/physics (Porter \& Ivie 2019).
            \item Build relevance by centering cultural sustainability in design decisions. E.g., \url{https://www.openscied.org/instructional-materials/8-4-earth-in-space/}
        \end{enumerate}
\end{enumerate}


{\scriptsize
\bibliography{biblio}
}
\end{document}
